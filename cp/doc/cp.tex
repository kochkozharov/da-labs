\documentclass[12pt]{article}

\usepackage{listings}
\usepackage{fullpage}
\usepackage{multicol,multirow}
\usepackage{tabularx}
\usepackage{ulem}
\usepackage[utf8]{inputenc}
\usepackage[russian]{babel}
\usepackage{amsmath}
\usepackage{pgfplots}

\begin{document}

\section*{Лабораторная работа №\,5 по курсу дискрeтного анализа: Динамическое программирование}

Выполнил студент группы М80-308Б-22 МАИ \textit{Кочкожаров Иван}.

\subsection*{Условие}

Краткое описание задачи:
\begin{enumerate}
    \item При помощи структуры данных суффиксное дерево разработать алгоритм решения задачи, определяемой
    своим вариантом
    \item \textit{Вариант:} Упрощенный вариант
    \item \textit{Задача:} Реализовать поиск подстрок в тексте с использование суффиксного дерева. Суффиксное дерево можно построить за $O(n^2)$ наивным методом. 
\end{enumerate}

\subsection*{Метод решения}

Наивный способ представляет собой compact trie, который построен по всем суффиксам
текущей строки, сначала помещается несобственный суффикс S[1..m]\$, затем
последовательно вводятся суффиксы S[i..m]\$ для i от 2 до m.
При этом мы должны сохранить свойства суффиксного дерева:
\begin{itemize}
    \item Trie должен оставаться compact trie
    \item В наивном алгоритме при вставке очередного суффикса сравниваем первую букву
\end{itemize}
Количество листьев в trie совпадает с количеством суффиксов в строке
суффикса и детей корня, если совпадений нет, то есть из корня мы не можем перейти
ни по одному из ребер, то вставляем весь суффикс от корня. Иначе, идем по совпавшему
ребру до того момента, пока встретим несовпадение. Тогда нам будет необходимо сделать
split текущего ребра, то есть уменьшить его правую границу, создать 2 новых ребра,
одно из которых будет отвечать за продолжение исходной дуги, ему необходимо будет
перекопировать детей, а второе будет являться оставшейся частью суффикса. Если мы
дошли до конца ребра и не встретили несовпадений, то ищем среди детей этого ребра
букву, по которой можем перейти дальше, если общая буква существует, то повторяем
алгоритм заново, если нет, то вставляем оставшийся суффикс от текущего ребра.
Поиск паттерна работает аналогично вставке, только когда мы встречаем несовпадение,
то просто останавливаем цикл, если же мы не встретили несовпадений и при этом
паттерн закончился, то мы нашли вхождение. Чтобы узнать на каких позициях есть
вхождение, необходимо от текущей вершины обойти всех детей поиском в глубину. Так
мы получим массив, в котором будут хранится индексы вхождений паттерна в текст.

\subsection*{Описание программы}

Разделение по файлам, описание основных типов данных и функций. \\
\begin{itemize}
    \item
        suffix\_tree.h,
        \begin{lstlisting}[language=C++]
#pragma once 

#include <memory>
#include <map>
#include <vector>
#include <string>

class SuffixTrie {
private:
    struct Node {
        size_t l;
        size_t r;
        size_t enter;
        std::map<char, std::shared_ptr<Node>> child;
    };

    std::shared_ptr<Node> root;
    std::string text;

    void Insert(size_t l, size_t r);
    static std::shared_ptr<Node> NewNode
            (size_t l, size_t r, size_t enter);
    void DFS(std::vector<size_t>& res,
        std::shared_ptr<Node> current);

public:
    explicit SuffixTrie(const std::string& inputText);
    std::vector<size_t> Search(const std::string& pattern);
};
        \end{lstlisting}
        \item
        suffix\_tree.cpp
        \begin{lstlisting}[language=C++]
#include "suffix_tree.h"

void SuffixTrie::Insert(size_t l, size_t r) {
    auto current = root;
    size_t oldL = l;

    while (l <= r) {
        if (current == root) {
            if (current->child.find(text[l]) 
                != current->child.end()) {
                current = current->child[text[l]];
            } else {
                current->child[text[l]] = NewNode(l, r, oldL);
                break;
            }
        }

        size_t start = current->l;
        size_t finish = current->r;
        bool cut = false;

        for (size_t i = start; (i <= finish) &&
             (l + i - start <= r); ++i) {
            if (text[i] != text[l + i - start]) {
                current->r = i - 1;
                size_t oldEnter = current->enter;

                auto splitNode = NewNode(i, finish, oldEnter);
                splitNode->child = move(current->child);
                current->child.clear();
                current->child[text[i]] = splitNode;

                current->child[text[l + i - start]] 
                = NewNode(l + i - start, r, oldL);
                cut = true;
                break;
            }
        }

        if (!cut) {
            size_t newL = l + finish - start + 1;
            if (current->child.find(text[newL]) !=
                 current->child.end()) {
                current = current->child[text[newL]];
            } else {
                current->child[text[newL]] =
                     NewNode(newL, r, oldL);
                break;
            }
            l = newL;
        } else {
            break;
        }
    }
}


std::shared_ptr<SuffixTrie::Node> SuffixTrie::NewNode
                    (size_t l, size_t r, size_t enter) {
    auto vertex = std::make_shared<Node>();
    vertex->l = l;
    vertex->r = r;
    vertex->enter = enter;
    return vertex;
}

void SuffixTrie::DFS(std::vector<size_t> &res, 
            std::shared_ptr<SuffixTrie::Node> current) {
    if (current->child.empty()) {
        res.push_back(current->enter);
    }
    for (const auto &[first, child] : current->child) {
        DFS(res, child);
    }
}

SuffixTrie::SuffixTrie(const std::string &inputText) {
    text = inputText + '$';
    root = std::make_shared<Node>();
    for (size_t i = 0; i < text.length(); ++i)
        Insert(i, text.length() - 1);
}

std::vector<size_t> SuffixTrie::Search
            (const std::string &pattern) {
    std::vector<size_t> res;
    auto current = root;
    size_t l = 0;
    size_t r = pattern.length() - 1;
    bool flag = false;

    while (l <= r) {
        if (current == root) {
            if (current->child.find(pattern[l]) !=
                             current->child.end()) {
                current = current->child[pattern[l]];
            } else {
                break;
            }
        }

        size_t start = current->l;
        size_t finish = current->r;
        for (size_t i = 0; (start + i <= finish) &&
                                     (i + l <= r); ++i) {
            if (pattern[i + l] != text[start + i]) {
                flag = true;
                break;
            }
        }

        if (!flag) {
            l = l + finish - start + 1;
            if (l > r) {
                break;
            }
            if (current->child.find(pattern[l]) !=
                                 current->child.end()) {
                current = current->child[pattern[l]];
            } else {
                break;
            }
        } else {
            break;
        }
    }

    if ((l > r) && (!flag) && (!pattern.empty())) {
        DFS(res, current);
    }
    return res;
}
        \end{lstlisting}
        \item
        main.cpp
        \begin{lstlisting}[language=C++]
#include <algorithm>
#include <iostream>
#include "suffix_tree.h"

using namespace std;

int main() {
    string text;
    cin >> text;
    SuffixTrie SuffixTrie(text);

    string pattern;
    size_t counter = 1;
    while (cin >> pattern) {
        vector<size_t> res = SuffixTrie.Search(pattern);
        if (!res.empty()) {
            printf("%zu: ", counter);
            sort(res.begin(), res.end());
            for (size_t i = 0; i < res.size(); ++i) {
                if (i != 0) {
                    printf(", ");
                }
                printf("%zu", res[i] + 1);
            }
            printf("\n");
        }
        ++counter;
    }

    return 0;
}
        \end{lstlisting}
\end{itemize}

Функции-члены класса:
Приватные:
\begin{enumerate}
    \item Insert - вставляет часть строки с индекса l по индекс r в trie
    \item NewNode - статический метод для создания узла без детей
    \item DFS - функция, которая обходит всех детей узла и сохраняет номера листьев переданный вектор
\end{enumerate}
Публичные
\begin{enumerate}
    \item SuffixTrie - конструктор, принимающий одну строку. Он добавляет сентинел к строке и всталяет все преффиксы строки в trie c помощью Insert
    \item Search - возвращает вектор с индексами всех вхождений строки в текст
\end{enumerate}

\subsection*{Дневник отладки}

Во время реализации я столкнулся с проблемами:
\begin{enumerate}
    \item Проблема с алгоритмом. Изначально на чекере выдывало Memory Limit, для избежания проблемы необходимо было использовать сжатое суффиксное дерево. Поэтому пришлось полностью переписать алгоритм.
\end{enumerate}

\subsection*{Тест производительности}

\begin{lstlisting}[language=C++]
#include "suffix_tree.h"
#include <cstddef>
#include <cstdio>
#include <iostream>
#include <string>
#include <vector>
#include <cassert>
#include <chrono>
#include <algorithm>
#include <random>

using duration_t = std::chrono::microseconds;

using namespace std;

string generateRandomString(int length)
{
    // Define the list of possible characters
    const string CHARACTERS
        = "abcdefghijklmnopqrstuvwxyz";

    // Create a random number generator
    random_device rd;
    mt19937 generator(rd());

    // Create a distribution to uniformly select from all
    // characters
    uniform_int_distribution<> distribution(
        0, CHARACTERS.size() - 1);

    // Generate the random string
    string random_string;
    for (int i = 0; i < length; ++i) {
        random_string
            += CHARACTERS[distribution(generator)];
    }

    return random_string;
}

int main() {

    string text=generateRandomString(10000);
    long t1=0;
    long t2=0;
    auto start = std::chrono::system_clock::now();
    SuffixTrie SuffixTrie(text); 
    auto end = std::chrono::system_clock::now();
    t1 += std::chrono::duration_cast<duration_t>
        (end - start).count();

    for (int i = 0; i < 10000; ++i) {
        string pattern = generateRandomString(10);
        auto start = std::chrono::system_clock::now();
        vector<size_t> res1 = SuffixTrie.Search(pattern);
        auto end = std::chrono::system_clock::now();
        t1 += std::chrono::duration_cast<duration_t>
            (end - start).count();


        vector<size_t> res2;
        start = std::chrono::system_clock::now();
        size_t pos = text.find(pattern, 0);
        while (pos != string::npos) {
            res2.push_back(pos);
            pos = text.find(pattern, pos + 1);
        }
        end = std::chrono::system_clock::now();
        t2 += std::chrono::duration_cast<duration_t>
            (end - start).count();
        sort(res1.begin(), res1.end());
        sort(res2.begin(), res2.end());


        assert(res1 == res2);
    }
    cout << "suf_tree: " << t1 
        << '\n' << "naive: "  << t2 << '\n';
}
\end{lstlisting}

Длинна текста будет фиксированной и равной 10000. Попробуем искать все вхождения N строки из 10 символов в этом тексте
используя последловательный std::find и используя suffix\_trie


\begin{tikzpicture}
    \begin{axis} [
        ymin = 0,
        ylabel=$ms$,
        xlabel=$N$,
    ]
    \addplot coordinates {
         (10000, 69046) (20000,207233) (30000,414403) (40000,690395) (50000,1035087)
    };
    \addplot coordinates {
         (10000, 28171) (20000,42338) (30000,61732) (40000,85091) (50000,116014)
    };
    \end{axis}
\end{tikzpicture}

Таким образом видим, что суффиксное дерево отлично подходит для поиска большого количество паттернов в одном неизменном тексте.
Поиск одного паттерна через std::find имеет сложность $O(n \times k + m)$ где k - количество вхождений паттерна,
а поиск через суффиксное дерево $O(m+k)$. $O(m)$ на спуск и $O(k)$ на обход поддерева.

\subsection*{Выводы}

В этой лабораторной работе я познакомился с практическим применением суффиксного
дерева. В отличие от других алгоритмов поиска подстроки в строки, суффиксное дерево
позволяет обработать текст, а не паттерны. Я научился строить сжатое суффиксное дерево
наивным методом и выполнять в нем поиск. Теперь я знаю как работает ctrl+f (поиск по странице) в браузерах.

\end{document}