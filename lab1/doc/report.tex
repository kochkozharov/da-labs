\documentclass[12pt]{article}

\usepackage{listings}
\usepackage{fullpage}
\usepackage{multicol,multirow}
\usepackage{tabularx}
\usepackage{ulem}
\usepackage[utf8]{inputenc}
\usepackage[russian]{babel}

\begin{document}

\section*{Лабораторная работа №\,1 по курсу дискрeтного анализа: сортировка за линейное время}

Выполнил студент группы М80-208Б-22 МАИ \textit{Кочкожаров Иван}.

\subsection*{Условие}

Краткое описание задачи:
\begin{enumerate}
\item
    Требуется разработать программу, осуществляющую ввод пар "ключ-значение", сортировку по возрастанию ключа указанным алгоритмом сортировки за линейное время и вывод получившейся последовательности.
\item
    Вариант задания:
        \begin{itemize}
            \item Карманная сортировка.
            \item Тип ключа: вещественные числа от -100 до 100.
            \item Тип значения: строки фиксированной длины 64 символа, во входных данных могут встретиться строки меньшей длины, при этом строка дополняется до 64-х нулевыми символами, которые не выводятся на экран.
        \end{itemize}
\end{enumerate}

\subsection*{Метод решения}

Неизвестно, сколько конкретно пар "ключ-значение" будет подано в программу, поэтому необходимо реализовать динамический массив, он же вектор, способный менять свой размер для хранения данных.
\par Далее требуется реализовать стабильную камранную сортировку. Ее суть заключается в том, что определяеятся функция, которая ставит в соответсвие элементам массива номер кармана, в который он будет помещен.
элементы все элементы в i-ом кармане должны быть меньше, чем все элементы в i+1-ом кармане. Затем каждый карман сортируется стабильной сравнительной сортировкой (выбрана сортировка слиянием) и полученный карманы конкатенируются.
Если входные элементы подчиняются равномерному закону распределения, то сортировка работает за O(n).

\subsection*{Описание программы}

Разделение по файлам, описание основных типов данных и функций. \\
\begin{itemize}
    \item
        Класс TVector. Реализованы конструкторы, основные методы, также перегружены некоторые операторы вроде индексатора.
        \begin{lstlisting}[language=C++]
template <class T>
class TVector {
    private:
    T* data{nullptr};
    std::size_t capacity{0}, size{0};
    static constexpr std::size_t NewCapacity(std::size_t capacity);
    public:
    void Reserve(std::size_t newCapacity);
    void PushBack(const T& value);
    void PushBack(T&& value);
    [[nodiscard]] constexpr std::size_t Size() const { return size; }
    [[nodiscard]] constexpr std::size_t Capacity() const { return size; }
    TVector() = default;
    TVector(std::size_t size) : data{new T[size]}, capacity{size} {}
    TVector(const TVector& other);
    TVector(TVector&& other) noexcept;
    TVector& operator=(const TVector& other);
    TVector& operator=(TVector&& other) noexcept;
    T operator[](std::size_t idx) const { return data[idx]; }
    T& operator[](std::size_t idx) { return data[idx]; }
    virtual ~TVector() noexcept { delete[] data; }
};
        \end{lstlisting}
    \item
     Структура TKeyValuePair. Необходима для хранения пар "ключ-значение".
        \begin{lstlisting}[language=C++]
struct TFixedSizeLine {
    inline static const int SIZE = 64;
    char str[SIZE + 1];
    TFixedSizeLine() = default;
    TFixedSizeLine(const char* a) { strcpy(str, a); }
    operator const char*() { return str; }
};
        
    class TKeyValuePair {
public:
    double key;
    TFixedSizeLine* value;
    TKeyValuePair();
    TKeyValuePair(int key, const char* value);
    TKeyValuePair(const TKeyValuePair& other);
    TKeyValuePair& operator=(const TKeyValuePair& other);
    TKeyValuePair& operator=(TKeyValuePair&& other);
    TKeyValuePair(TKeyValuePair&& other);

    ~TKeyValuePair();
    void Print(FILE* stream);
    bool Scan(FILE* stream);
}
        \end{lstlisting}
    \item
        Реализация карманной сортировки
        \begin{lstlisting}[language=C++]

void BucketSort(TVector<TKeyValuePair>& arr) {
    const int minElement = -100;
    const int maxElement = 100;
    const int range = maxElement - minElement;
    const size_t numBuckets = range;
    TVector<TVector<TKeyValuePair>> buckets(numBuckets);
    for (size_t i = 0; i < arr.Size(); ++i) {
        int bucketIndex = (arr[i].key - minElement) * (numBuckets - 1) / range;
        buckets[bucketIndex].PushBack(std::move(arr[i]));
    }

    size_t cnt = 0;
    TVector<TKeyValuePair> buf(arr.Size());
    for (size_t i = 0; i < numBuckets; ++i) {
        MergeSort(buckets[i], buf);
        for (size_t j = 0; j < buckets[i].Size(); ++j) {
            arr[cnt++] = std::move(buckets[i][j]);
        }
    }
}
        \end{lstlisting}
\end{itemize}

\subsection*{Дневник отладки}

Изначально в качестве вспомгательного алгоритма использовалась сортировка вставками, но только после замены ее на сортировку слиянием решения стало проходить чекер.
Так же сильно улучшило производительность хранение строки из 64 символов не в самом элементе вектора, а в куче. 

\subsection*{Тест производительности}

Померить время работы кода лабораторной и теста производительности
на разных объемах входных данных. Сравнить результаты. Проверить,
что рост времени работы при увеличении объема входных данных
согласуется с заявленной сложностью.

Карманная сортировка работает за линейное время. Для большей наглядности приведём таблицу, в которой написанная сортировка сравнивается со стандартными функция языка C++.

\begin{center}
\begin{tabular}{ |c|c|c|c| }
    \hline
    Количество пар "ключ-значение" & BucketSort(), мс & MergeSort(), мс  \\
    \hline
    100000 & 32458 & 48899   \\
    200000 & 104580 & 142907   \\
    300000 & 212336 & 284310   \\
    400000 & 371465 & 461866  \\
    500000 & 535948 & 693257  \\
    600000 & 798625 & 1016245   \\
    700000 & 1025358 & 1328589   \\
    800000 & 1533539 & 1783103   \\
    900000 & 1692229 & 2142605   \\
    1000000 & 2043095 & 2654381  \\


    \hline
    \end{tabular}
\end{center}

Карманная сортировка работает быстрее MergeSort, и на больших объемах данных время ее работы становится пропорциональным 
количеству данных.

Ниже приведена программа benchmark.cpp, использовавшаяся для засечения времени работы функций:
\begin{lstlisting}[language=C++]
#include <chrono>
#include <iostream>
#include <random>

#include "header.h"

int main() {
    TVector<TKeyValuePair> data;
    TVector<TKeyValuePair> data2;

    std::random_device rd;
    std::mt19937 gen(rd());
    std::uniform_real_distribution<double> distr(-100, 100);
    for (size_t i = 100000; i <= 1000000; i+=100000) {
        const size_t numberOfElements = i;
        std::cout << numberOfElements << '\n';
        for (size_t i = 0; i < numberOfElements; ++i) {
            data.PushBack(TKeyValuePair(distr(gen), "test"));
            data2.PushBack(TKeyValuePair(distr(gen), "test"));
        }

        auto start = std::chrono::high_resolution_clock::now();
        BucketSort(data);
        auto end = std::chrono::high_resolution_clock::now();
        std::cout << "Bucket sort time: "
                    << std::chrono::duration_cast<std::chrono::microseconds>(
                            end - start)
                            .count()
                    << " microseconds" << std::endl;

        // Benchmarking std::sort
        TVector<TKeyValuePair> buf(data2.Size());
        start = std::chrono::high_resolution_clock::now();
        MergeSort(data2, buf);
        end = std::chrono::high_resolution_clock::now();
        std::cout << "merge sort: "
                    << std::chrono::duration_cast<std::chrono::microseconds>(
                            end - start)
                            .count()
                    << " microseconds" << std::endl;
    }

    return 0;
}

\end{lstlisting}

\subsection*{Выводы}

В ходе выполнения программы был реализован алгоритм карманной сортировки. Данная сортировка предполагает равномерное распрделение
входны данных, но позволяет сортировать вещественные числа за O(n), в отличие от сортировки подсчетом.
Помимо этого был получен опыт реализации шаблонных структур данных.

\end{document}