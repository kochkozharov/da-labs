\CWHeader{Лабораторная работа \textnumero 2}

\CWProblem{
Необходимо создать программную библиотеку, реализующую указанную структуру данных, на основе которой разработать программу-словарь. 
В словаре каждому ключу, представляющему из себя регистронезависимую последовательность букв английского алфавита длиной не более 256 символов, 
поставлен в соответствие некоторый номер, от $0$ до $2^{64} - 1$. Разным словам может быть поставлен в соответствие один и тот же номер.
Программа должна обрабатывать строки входного файла до его окончания. Каждая строка может иметь следующий формат:

\textbf{+ word 34} -- добавить слово \enquote{word} с номером 34 в словарь. 
Программа должна вывести строку \enquote{OK}, если операция прошла успешно, \enquote{Exist}, если слово уже находится в словаре.

\textbf{- word} -- удалить слово \enquote{word} из словаря. 
Программа должна вывести \enquote{OK}, если слово существовало и было удалено, \enquote{NoSuchWord}, если слово в словаре не было найдено.

\textbf{word} -- найти в словаре слово \enquote{word}. Программа должна вывести \enquote{OK: 34}, если слово было найдено; 
число, которое следует за \enquote{OK:} -- номер, присвоенный слову при добавлении. В случае, если слово в словаре не было обнаружено, 
нужно вывести строку \enquote{NoSuchWord}.

\textbf{! Save /path/to/file} -- сохранить словарь в бинарном компактном представлении на диск в файл, указанный парамером команды. 
В случае успеха, программа должна вывести \enquote{OK}, в случае неудачи выполнения операции, программа должна вывести описание ошибки (см. ниже).

\textbf{! Load /path/to/file} -- загрузить словарь из файла. Предполагается, что файл был ранее подготовлен при помощи команды Save. 
В случае успеха, программа должна вывести строку \enquote{OK}, а загруженный словарь должен заменить текущий (с которым происходит работа); 
в случае неуспеха, должна быть выведена диагностика, а рабочий словарь должен остаться без изменений. Кроме системных ошибок, 
программа должна корректно обрабатывать случаи несовпадения формата указанного файла и представления данных словаря во внешнем файле.

Для всех операций, в случае возникновения системной ошибки (нехватка памяти, отсутсвие прав записи и т.п.), программа должна вывести строку, 
начинающуюся с \enquote{ERROR:} и описывающую на английском языке возникшую ошибку.

Различия вариантов заключаются только в используемых структурах данных.

{\bfseries Вариант структуры данных:} PATRICIA.
}
\pagebreak
