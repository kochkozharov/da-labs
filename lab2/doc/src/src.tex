\section{Описание}
Требуется написать реализацию PATRICIA Trie.

Как сказано в \cite{Mehta}: \enquote{Particia -- сжатый бинарный trie, в котором ветка и узлы элементов объединены в единый узел}.
Каждый узел помимо сохраняемых данных хранит номер бита, который будет проверяться у ключа во время поиска, и два указателя:
на левого и правого ребёнка.

\pagebreak

\section{Исходный код}
Так как словарь должен хранить пару \enquote{ключ-значение}, создадим структуру \textit{TPair}. Для узлов дерева в классе \textit{TPatriciaTrie}
определим приватную структуру \textit{TNode}, которая будет содержать поля \textit{data, children, bitNumber, id}.
Так же определим структуру \textit{TTCaseInsensitiveString}, которая будет содержать в себе регистронезависимую строку фиксированного размера.
Для удобства переопределим в классе структуру типа \textit{TPair<TTCaseInsensitiveString, uint64\_t>} как \textit{TData}.

Определим битовые операции со строками. Функция \textit{bool GetBitByIndex(const std::string\& str, int index)} возвращает
значение бита с номером \textit{index} в строке \textit{str}, используя двоичную маску.
Функция \textit{int GetBitDifference(const std::string\& a, const std::string\& b)} возвращает первый слева номер бита, различного
для строк \textit{a} и \textit{b}.

Реализуем поиск по ключу в дереве. Напишем приватный вспомогательный метод 
\textit{TPair<TPatriciaTrie::TNode*, int> FindPreviousNode(const TTCaseInsensitiveString\& key, int bitNumber)}.
Он возвращает пару, содержащую указатель на предыдущий узел (родитель искомого) и значение $n$-ного бита, которое необходимо для перехода к 
искомому узлу. Функция осуществляет обход дерева. В каждом узле вычисляется значения бита с номером \textit{bitNumber} в ключе. Если оно 
равно 0, то происходит переход по левому указателю, иначе -- по правому. Обход останавливается в момент, когда происходит переход по 
обратному указателю, то есть значение \textit{bitNumber} следующего меньше или равно значения \textit{bitNumber} предыдущего.

Приватный метод \textit{TNode*\& FindNode(const std::string\& key, int bitNumber)} и публичный 
\textit{const TData\& Find(const std::string\& key)} используют метод \textit{FindPreviousNode}.

Реализуем вставку в дерево -- метод \textit{bool Insert(const TData\& data)}. Алгоритм вставки
\textit{theKey} (взят из \cite{Mehta}):
\begin{enumerate}
	\item Найти \textit{theKey}. Пусть \textit{reachedKey} -- ключ узла, на котором поиск закончился.
	\item Определить первый бит слева \textit{lBitPos}, различный для \textit{theKey} и \textit{reachedKey}.
	\item Создать новый узел с ключом \textit{theKey}, в поле \textit{bitNumber} записать \textit{lBitPos}.
	Вставить этот узел между другими узлами, которые были пройдены во время поиска, так, чтобы последовательность 
	из \textit{bitNumber} была возрастающей. Эта вставка сломает указатель между двумя узлами $p$ и $q$.
	Указатель из $p$ теперь содержит новый узел.
	\item Если \textit{lBitPos} узла с ключом \textit{theKey} равен 1, то указатель на правого ребёнка становится 
	обратным указателем на этот узел. Иначе -- указатель на левого ребёнка становится обратным. Оставшийся
	указатель будет содержать $q$.
\end{enumerate}

Реализуем удаление из дерева -- метод \textit{void Erase(const std::string\& key)}. Пусть узел $p$ -- узел, который
мы хотим удалить. Возможны два случая:
\begin{enumerate}
	\item У $p$ есть указатель на самого себя. Если $p$ -- корень, то узел удаляется, а дерево становится пустым.
	Иначе -- указатель на $p$ его родителя устанавливаем к несобственному (который не указывает на самого себя) указателю.
	\item У $p$ нет указателя на самого себя. Ищем узел $q$, у которого есть обратный указатель на $p$. Данные, которые 
	хранятся в $q$ перемещаются в $p$, и мы удаляем $q$. Чтобы удалить $q$, нужно найти узел $r$, у которого есть
	обратный указатель на $q$. Обратный указатель на $q$ изменяем, чтобы он указывал на $p$. Прямой указатель от родителя
	$q$ изменяем, чтобы он указывал на ребёнка $q$.
\end{enumerate}

Метод \textit{void SaveToFile(FILE* file) const} сохраняет дерево в файл. Для этого сначала все узлы дерева записываются
в массив с помощью метода \textit{void TreeToArray(TNode** array, TNode* root, int\& id) const}. Все узлы, при этом, пронумерованы (поле \textit{id}).
Далее в файл записывается количество узлов, затем все узлы последовательно.

Метод \textit{void LoadFromFile(FILE* file)} загружает дерево из файла. Для этого все узлы из файла считываются, а потом с помощью
метода \textit{void ArrayToTree(TSaveData* array)} строится дерево. Указатели восстанавливаются с помощью \textit{id}.

\begin{longtable}{|p{7.5cm}|p{7.5cm}|}
\hline
\rowcolor{lightgray}
\multicolumn{2}{|c|} {binary\_string.h}\\
\hline
int GetBitSize(const std::string\& str)&Функция получения количества бит в строке.\\
\hline
int GetBitDifference(const std::string\& a, const std::string\& b)&Функция получения индекса различного бита.\\
\hline
bool GetBitByIndex(const std::string\& str, int index)&Функция получения значения бита по индексу.\\
\hline
\end{longtable}

\begin{lstlisting}[language=C++]

struct TFile {
    enum class FileType { Save, Load };
    FILE* file;
    TCaseInsensitiveString name;
    TFile(TCaseInsensitiveString s, FileType mode) {
        name = s;
        if (mode == FileType::Save) {
            file = fopen(s.CStr(), "wb");
        } else {
            file = fopen(s.CStr(), "rb");
        }

    }
    bool check() {
        return file != nullptr;
    }
    FILE* GetFile() { return file; }
    TFile() = delete;
    TFile(const TFile& other) = delete;
    TFile(TFile&& other) = delete;
    TFile& operator=(const TFile& other) = delete;
    TFile& operator=(TFile&& other) = delete;
    ~TFile() { fclose(file); }
};

template <class T, class U>
struct TPair {
    T key;
    U value;

    TPair() = default;
    TPair(const T& key, const U& value) : key(key), value(value) {}
    TPair(T&& key, U&& value) : key(key), value(value) {}
    TPair(const TPair& other) : key(other.key), value(other.value) {}
    TPair(TPair&& other) noexcept
        : key(std::move(other.key)), value(std::move(other.value)) {}
    TPair& operator=(const TPair& other) {
        key = other.key;
        value = other.value;
        return *this;
    }
    TPair& operator=(TPair&& other) noexcept {
        key = std::move(other.key);
        value = std::move(other.value);
        return *this;
    }
};

class TPatriciaTrie {
   private:
    using TData = TPair<TCaseInsensitiveString, uint64_t>;

    struct TNode {
        TData data;
        TNode* children[2];
        int bitNumber;
        int id;

        TNode() = default;
        TNode(const TData& data);
    };

    struct TSaveData {
        int id;
        char key[257];
        uint64_t value;
        int bitNumber;
        int leftId;
        int rightId;
    };

    TNode* root;
    int size;

    TNode*& FindNode(const TCaseInsensitiveString& key, int bitNumber);
    TPair<TPatriciaTrie::TNode*, int> FindPreviousNode(
        const TCaseInsensitiveString& key, int bitNumber);
    void DestroyTrie(TNode* node);
    void TreeToArray(TNode** array, TNode* root, int& id) const;
    void ArrayToTree(TSaveData* array);

   public:
    TPatriciaTrie();
    ~TPatriciaTrie();
    bool Insert(const TData& data);
    const TData *Find(const TCaseInsensitiveString& key);
    bool Erase(const TCaseInsensitiveString& key);
    bool SaveToFile(FILE* file) const;
    bool LoadFromFile(FILE* file);
    int Size() const;
};
\end{lstlisting}
\pagebreak

\section{Консоль}
\begin{alltt}
ivan@asus-vivobook ~/c/d/b/lab2 (master)> cat testcase
+ word 444
+ word 3
+ wordy 666
! Save patricia.dat
+ wo 555
! Load patricia.dat
word
- word
wo
ivan@asus-vivobook ~/c/d/b/lab2 (master)> ./lab2 < testcase
OK
Exist
OK
OK
OK
OK
OK: 444
OK
NoSuchWord
\end{alltt}
\pagebreak
