\section{Тесты}

Тесты написаны с использованием библиотеки \textit{gtest} 

\begin{lstlisting}[language=C++]

TEST(patricia_test, modifier01) {
    TPatriciaTrie p;
    p.Insert({"a", 1});
    EXPECT_EQ(p.Find("a")->value, 1);
    EXPECT_EQ(p.Size(), 1);
}

TEST(patricia_test, modifier02) {
    TPatriciaTrie p;
    p.Insert({"a", 1});
    EXPECT_FALSE(p.Insert({"A", 2}));
}

TEST(patricia_test, modifier03) {
    TPatriciaTrie p;
    p.Insert({"abc", 10});
    EXPECT_EQ(p.Size(), 1);
    EXPECT_FALSE(p.Insert({"abc", 20}));
}

TEST(patricia_test, modifier04) {
    TPatriciaTrie p;
    p.Insert({"ab", 10});
    p.Insert({"abc", 20});
    EXPECT_EQ(p.Size(), 2);
    EXPECT_EQ(p.Find("ab")->value, 10);
    EXPECT_EQ(p.Find("abc")->value, 20);
}
\end{lstlisting}

Вывод в консоль после запуска тестов через valgrind

\begin{alltt}
    ==149268== Memcheck, a memory error detector
    ==149268== Copyright (C) 2002-2022, and GNU GPL'd, by Julian Seward et al.
    ==149268== Using Valgrind-3.21.0 and LibVEX; rerun with -h for copyright info
    ==149268== Command: ./lab2_tests
    ==149268== 
    Running main() from /var/tmp/portage/dev-cpp/gtest-1.13.0/work/
    googletest-1.13.0/googletest/src/gtest_main.cc
    [==========] Running 38 tests from 2 test suites.
    [----------] Global test environment set-up.
    [----------] 2 tests from binary_string_test
    [ RUN      ] binary_string_test.bitdifftest01
    [       OK ] binary_string_test.bitdifftest01 (8 ms)
    [ RUN      ] binary_string_test.bitdifftest02
    [       OK ] binary_string_test.bitdifftest02 (1 ms)
    [----------] 2 tests from binary_string_test (13 ms total)
    
    [----------] 36 tests from patricia_test
    [ RUN      ] patricia_test.modifier01
    [       OK ] patricia_test.modifier01 (5 ms)
    [ RUN      ] patricia_test.modifier02
    [       OK ] patricia_test.modifier02 (3 ms)
    [ RUN      ] patricia_test.modifier03
    [       OK ] patricia_test.modifier03 (2 ms)
    [ RUN      ] patricia_test.modifier04
    [       OK ] patricia_test.modifier04 (3 ms)
    [ RUN      ] patricia_test.modifier05
    [       OK ] patricia_test.modifier05 (2 ms)
    [ RUN      ] patricia_test.modifier06
    [       OK ] patricia_test.modifier06 (4 ms)
    [ RUN      ] patricia_test.modifier07
    [       OK ] patricia_test.modifier07 (3 ms)
    [ RUN      ] patricia_test.modifier08
    [       OK ] patricia_test.modifier08 (1 ms)
    [ RUN      ] patricia_test.modifier09
    [       OK ] patricia_test.modifier09 (1 ms)
    [ RUN      ] patricia_test.modifier10
    [       OK ] patricia_test.modifier10 (4 ms)
    [ RUN      ] patricia_test.modifier11
    [       OK ] patricia_test.modifier11 (3 ms)
    [ RUN      ] patricia_test.modifier12
    [       OK ] patricia_test.modifier12 (7 ms)
    [ RUN      ] patricia_test.modifier13
    [       OK ] patricia_test.modifier13 (6 ms)
    [ RUN      ] patricia_test.modifier14
    [       OK ] patricia_test.modifier14 (6 ms)
    [ RUN      ] patricia_test.modifier15
    [       OK ] patricia_test.modifier15 (1867 ms)
    [ RUN      ] patricia_test.modifier16
    [       OK ] patricia_test.modifier16 (1209 ms)
    [ RUN      ] patricia_test.modifier17
    [       OK ] patricia_test.modifier17 (1147 ms)
    [ RUN      ] patricia_test.modifier18
    [       OK ] patricia_test.modifier18 (8 ms)
    [ RUN      ] patricia_test.insert01
    [       OK ] patricia_test.insert01 (1 ms)
    [ RUN      ] patricia_test.insert02
    [       OK ] patricia_test.insert02 (2 ms)
    [ RUN      ] patricia_test.insert03
    [       OK ] patricia_test.insert03 (3 ms)
    [ RUN      ] patricia_test.insert04
    [       OK ] patricia_test.insert04 (1 ms)
    [ RUN      ] patricia_test.erase01
    [       OK ] patricia_test.erase01 (1 ms)
    [ RUN      ] patricia_test.erase02
    [       OK ] patricia_test.erase02 (2 ms)
    [ RUN      ] patricia_test.erase03
    [       OK ] patricia_test.erase03 (2 ms)
    [ RUN      ] patricia_test.erase04
    [       OK ] patricia_test.erase04 (5 ms)
    [ RUN      ] patricia_test.erase05
    [       OK ] patricia_test.erase05 (5 ms)
    [ RUN      ] patricia_test.erase06
    [       OK ] patricia_test.erase06 (5 ms)
    [ RUN      ] patricia_test.erase07
    [       OK ] patricia_test.erase07 (4 ms)
    [ RUN      ] patricia_test.erase08
    [       OK ] patricia_test.erase08 (3 ms)
    [ RUN      ] patricia_test.erase09
    [       OK ] patricia_test.erase09 (4 ms)
    [ RUN      ] patricia_test.erase10
    [       OK ] patricia_test.erase10 (4 ms)
    [ RUN      ] patricia_test.file01
    [       OK ] patricia_test.file01 (20 ms)
    [ RUN      ] patricia_test.file02
    [       OK ] patricia_test.file02 (7 ms)
    [ RUN      ] patricia_test.file03
    [       OK ] patricia_test.file03 (4 ms)
    [ RUN      ] patricia_test.file04
    [       OK ] patricia_test.file04 (1189 ms)
    [----------] 36 tests from patricia_test (5568 ms total)
    
    [----------] Global test environment tear-down
    [==========] 38 tests from 2 test suites ran. (5610 ms total)
    [  PASSED  ] 38 tests.
    ==149268== 
    ==149268== HEAP SUMMARY:
    ==149268==     in use at exit: 311,335 bytes in 5 blocks
    ==149268==   total heap usage: 2,043 allocs, 2,038 frees, 928,529 bytes allocated
    ==149268== 
    ==149268== LEAK SUMMARY:
    ==149268==    definitely lost: 0 bytes in 0 blocks
    ==149268==    indirectly lost: 0 bytes in 0 blocks
    ==149268==      possibly lost: 0 bytes in 0 blocks
    ==149268==    still reachable: 311,335 bytes in 5 blocks
    ==149268==         suppressed: 0 bytes in 0 blocks
    ==149268== Rerun with --leak-check=full to see details of leaked memory
    ==149268== 
    ==149268== For lists of detected and suppressed errors, rerun with: -s
    ==149268== ERROR SUMMARY: 0 errors from 0 contexts (suppressed: 0 from 0)
\end{alltt}

Этот вывод valgrind говорит о том, что в программе нет утечек памяти.

\pagebreak