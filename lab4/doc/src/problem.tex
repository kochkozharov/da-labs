\CWHeader{Лабораторная работа \textnumero 4}

\CWProblem{
Необходимо реализовать один из стандартных алгоритмов поиска образцов для указанного алфавита.

\textbf{Вариант алгоритма} -- Поиск одного образца основанный на построении Z-блоков.

\textbf{Вариант алфавита} -- Числа в диапазоне от 0 до $2^{32} - 1$. Запрещается реализовывать алгоритмы на алфавитах меньшей размерности, чем указано в задании.

\textbf{Формат ввода}
Искомый образец задаётся на первой строке входного файла.
В случае, если в задании требуется найти несколько образцов, они задаются по одному на строку вплоть до пустой строки.
Затем следует текст, состоящий из слов или чисел, в котором нужно найти заданные образцы.
Никаких ограничений на длину строк, равно как и на количество слов или чисел в них, не накладывается.

\textbf{Формат вывода}
В выходной файл нужно вывести информацию о всех вхождениях искомых образцов в обрабатываемый текст: по одному вхождению на строку.
Для заданий, в которых требуется найти только один образец, следует вывести два числа через запятую: номер строки и номер слова в строке, с которого начинается найденный образец. В заданиях с большим количеством образцов, на каждое вхождение нужно вывести три числа через запятую: номер строки; номер слова в строке, с которого начинается найденный образец; порядковый номер образца.
Нумерация начинается с единицы. Номер строки в тексте должен отсчитываться от его реального начала (то есть, без учёта строк, занятых образцами).
Порядок следования вхождений образцов несущественен.
}

\pagebreak
