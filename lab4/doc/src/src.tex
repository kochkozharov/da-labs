\section{Описание}
Требуется написать реализацию Z-функции.

Как сказано в \cite{Gusfield}: \enquote{$Z_i(S)$ -- это длина наибольшего префикса $S[i..|S|]$, совпадающего с префиксом S}.

\section{Исходный код}
Будем хранить индексы L и R, обозначающие начало и конец префикса с наибольшим найденным на данный момент значением R. Изначально L = R = 0. 
Пусть нам известны значения Z-функции для позиций $1..i - 1$. Попробуем вычислить значение Z-функции для позиции i. Если $i \in [L..R]$, 
рассмотрим значение Z-функции для позиции j = i - L. Если $i + Z [ j ] \leq R$, то $Z [ i ] = Z [ j ]$, так как мы находимся в подстроке,
совпадающей с префиксом всей строки. Если же i + Z [ j ]  R, то необходимо досчитать значение Z[i] простым циклом, перебирающим символы 
после R, пока не найдется символ, не совпадающий с соответствующим символом из префикса. После этого изменяем, значение L на i и значение 
R на номер последнего символа, совпавшего с соответствующим символом из префикса.

Если $i \notin [L..R]$, то считаем значение Z[i] простым циклом, сравнивающим символы подстроки начинающейся с i-го
символа и соответствующие символы из префикса. Когда будет найдено несоответствие или будет достигнут конец строки, изменяем значение L 
на i и значение R на номер последнего символа, совпавшего с соответствующим символом из префикса. 

Реализация самого алгоритма z-блоков.

Образуем строку s = pattern + \# + text, где \# — символ, не встречающийся ни в text, ни в pattern. Вычисляем Z-функцию от этой строки.
В полученном массиве, в позициях в которых значение Z-функции равно |pattern|, по определению начинается подстрока, совпадающая с pattern. 

Полученные абсолютные индексы начала совпадающей подстроки переводятся в относительные коордианты вида номер строки и номер числа в начинающейся
используя массив префиксных сумм и бинарный поиск по нему. 

\section{Асимптотика}
Время работы алгоритма, вычисляющего значение Z-функции строки S оценивается в O (|S|). Докажем это.
Рассмотрим i-й символ строки. В алгоритме он рассматривается не более двух раз: первый раз, когда попадает в отрезок [L..R],
и второй раз при вычислении Z[i]. Таким образом цикл обрабатывает не более 2|S| итераций. 

\section{Листинги}

\begin{lstlisting}[language=C++]

using ll = long long;

std::size_t RelativeToAbsolute(const std::pair<size_t, size_t> &relativeIndex,
                               const std::vector<std::size_t> &prefixSums) {
    return prefixSums[relativeIndex.first] + relativeIndex.second;
}

std::pair<size_t, size_t>
AbsoluteToRelative(std::size_t absoluteIndex,
                   const std::vector<std::size_t> &prefixSums) {
    size_t lineIndex = std::lower_bound(prefixSums.begin(), prefixSums.end(),
                                        absoluteIndex + 1) -
                       prefixSums.begin();
    size_t numberIndex =
        absoluteIndex - (lineIndex > 0 ? prefixSums[lineIndex - 1] : 0) + 1;
    return std::make_pair(lineIndex, numberIndex);
}

std::vector<std::size_t> ZFunction(const std::vector<ll> &arr) {
    std::size_t n = arr.size();
    std::vector<std::size_t> z(arr.size());
    for (std::size_t i = 1, l = 0, r = 0; i < arr.size(); ++i) {
        if (i <= r) {
            z[i] = std::min(r - i + 1, z[i - l]);
        }
        while (i + z[i] < n && arr[z[i]] == arr[i + z[i]]) {
            ++z[i];
        }
        if (i + z[i] - 1 > r) {
            l = i;
            r = i + z[i] - 1;
        }
    }
    return z;
}

int main() {
    std::vector<ll> preparedText;
    std::string line;
    std::getline(std::cin, line);
    uint num;
    std::istringstream iss{line};
    std::size_t patternSize = 0;
    while (iss >> num) {
        patternSize++;
        preparedText.push_back(num);
    }
    preparedText.push_back(-1);
    std::vector<std::size_t> prefixSums(1);
    while (std::getline(std::cin, line)) {
        std::istringstream iss{line};
        std::size_t cnt = 0;
        while (iss >> num) {
            preparedText.push_back(num);
            cnt++;
        }
        prefixSums.push_back(prefixSums.back()+cnt);
    }

    std::vector<std::size_t> zFun = ZFunction(preparedText);
    for (std::size_t i = patternSize+1; i < zFun.size(); ++i) {
        if (zFun[i] == patternSize) {
            std::pair<size_t, size_t> match = AbsoluteToRelative(i-patternSize-1, prefixSums);
            std::cout << match.first << ", " << match.second;
            std::cout << '\n';
        }
    }
}
\end{lstlisting}

\section{Консоль}
\begin{alltt}
ivan@asus-vivobook ~/c/d/b/lab4 (master)> ./lab4
11 45 11 45 90
0011 45 011 0045 11 45 90    11
45 11 45 90
1, 3
1, 8
\end{alltt}
\pagebreak
