\documentclass[12pt]{article}

\usepackage{listings}
\usepackage{fullpage}
\usepackage{multicol,multirow}
\usepackage{tabularx}
\usepackage{ulem}
\usepackage[utf8]{inputenc}
\usepackage[russian]{babel}
\usepackage{amsmath}
\usepackage{pgfplots}


\begin{document}

\section*{Лабораторная работа №\,8 по курсу дискрeтного анализа: Жадные алгоритмы}

Выполнил студент группы М80-308Б-22 МАИ \textit{Кочкожаров Иван}.

\subsection*{Условие}

Краткое описание задачи:

Разработать жадный алгоритм решения задачи, определяемой своим
вариантом. Доказать его корректность, оценить скорость и объём
затрачиваемой оперативной памяти.

Реализовать программу на языке C или C++, соответствующую
построенному алгоритму. Формат входных и выходных данных описан в
варианте задания.

Бычкам дают пищевые добавки, чтобы ускорить их рост. Каждая добавка
содержит некоторые из $N$ действующих веществ. Соотношения количеств веществ в добавках могут отличаться. Воздействие добавки
определяется как

$c_1\times a_1 + c_2 \times a_2 +·\ldots +c_n \times a_n,$

где $a_i$ — количество $i$-го вещества в добавке, $c_i$ — неизвестный
коэффициент, связанный с веществом и не зависящий от добавки. Чтобы
найти неизвестные коэффициенты $c_i$, Биолог может измерить
воздействие любой добавки, использовав один её мешок. Известна цена
мешка каждой из $M$ ($M \geq N$ ) различных добавок. Нужно помочь Биологу
подобрать самый дешевый набор добавок, позволяющий найти
коэффициенты $c_i$. Возможно, соотношения веществ в добавках таковы,
что определить коэффициенты нельзя.

\subsection*{Метод решения}

Жадные алгоритмы применимы в том случае, если принятие наиболее 
оптимального решения на каждом шаге решения задачи означает наиболее 
оптимальное решение задачи в целом. К этой задаче можно применить жадный 
алгоритм, поскольку чтобы её решить, мы должны отобрать ровно $N$ добавок, 
а значит эти добавки должны быть наиболее дешёвыми. Идея решения в том, 
чтобы привести матрицу, составленную из соотношений веществ (для этого вызывается функция SubtractRows), к ступенчатому 
виду, при этом наверх продвигать строки, характеризующие наиболее дешёвые 
добавки (строчка ищется функцией FindLowestPriceRow), тогда $N$ верхних строк и будут ответом к задаче. Для получения 
ответа, необходимо сохранять номера добавок. Таким образом, алгоритм заключается в прохождении по всем добавкам,
вызывая функции FindLowestPriceRow и выдвижении найденой строчки вперед и вычитанию этой строчки из последующих, для приведения матрицы к ступенчатому виду

Итоговая сложность $O(M \times N + N^2 \times M) = O(N^2 \times M).$

\subsection*{Описание программы}

Разделение по файлам, описание основных типов данных и функций. \\

\begin{lstlisting}[language=C++]
#include <algorithm>
#include <iostream>
#include <vector>

constexpr const int MAX_NUM = 50;

struct Addition {
    Addition(size_t n) : ratios(n) {}
    std::vector<double> ratios;
    int price;
    int index;
};

int FindLowestPriceRow(std::vector<Addition> &v, int t) {
    size_t m = v.size();
    size_t n = v[0].ratios.size();
    int minPrice = MAX_NUM + 1;
    int index = -1;
    for (int i = t; i < m; ++i) {
        if ((v[i].ratios[t] != 0.0) && (v[i].price < minPrice)) {
            index = i;
            minPrice = v[i].price;
        }
    }
    return index;
}

void SubtractRows(std::vector<Addition> &v, int t) {
    size_t m = v.size();
    size_t n = v[0].ratios.size();
    for (int i = t + 1; i < m; ++i) {
        double coeff = v[i].ratios[t] / v[t].ratios[t];
        for (int j = t; j < n; ++j) {
            v[i].ratios[j] -= v[t].ratios[j] * coeff;
        }
    }
}

std::vector<int> Solve(std::vector<Addition> &additions) {
    size_t m = additions.size();
    size_t n = additions[0].ratios.size();
    std::vector<int> res;

    for (int i = 0; i < n; ++i) {
        int index = FindLowestPriceRow(additions, i);
        if (index == -1) {
            return {};
        }

        std::swap(additions[i], additions[index]);
        res.push_back(additions[i].index);
        SubtractRows(additions, i);
    }

    std::sort(res.begin(), res.end());
    return res;
}


int main() {
    int n, m;
    std::cin >> m >> n;
    std::vector<Addition> additions(m, Addition(n));

    for (int i = 0; i < m; ++i) {
        for (int j = 0; j < n; ++j) {
            std::cin >> additions[i].ratios[j];
        }
        std::cin >> additions[i].price;
        additions[i].index = i;
    }

    std::vector<int> res = Solve(additions);

    if (res.empty()) {
        std::cout << "-1\n";
        return 0;
    }

    for (auto r : res) {
        std::cout << r + 1 << ' ';
    }
    std::cout << '\n';
    return 0;
}
\end{lstlisting}

\subsection*{Дневник отладки}

\begin{enumerate}
    \item Было решено объединять все строки в матрицу и приводить ее к ступенчатому виду постепенно, вместо формирования матрицы для каждого набора строк и проверки его ранга.
    \item Изменены типы данных некоторых переменных
    \item Выполнена отладка некоторых функций, обнаружены ошибки в индексах
    матрицы
\end{enumerate}

\subsection*{Тест производительности}

Ниже приведен тест времени работы алгоритма. По оси $X$ — количество 
добавок при фиксированном количестве веществ, по оси $Y$ — время выполнения алгоритма в мс (меньше — лучше).
На втором графике изменяется количество веществ при фиксированном количестве добавок.

\begin{tikzpicture}
    \begin{axis} [
        ymin = 0,
        ylabel=$ms$,
        xlabel=$M$,
    ]
    \addplot coordinates {
        (100,1) (500,6) (1000,13) (5000, 62) (10000, 121) 
    };
    \end{axis}
\end{tikzpicture}


\begin{tikzpicture}
    \begin{axis} [
        ymin = 0,
        ylabel=$ms$,
        xlabel=$N$,
    ]
    \addplot coordinates {
        (50,12) (100,45) (200,176) (300, 366) 
    };
    \end{axis}
\end{tikzpicture}

Сложность алгоритма $O(N^2 \times M)$ показана графически

Ниже приведена программы, использовавштеся для генерации данных для теста:

\begin{lstlisting}[language=Python]
import random
import sys

file = open('test.txt', 'w')

m = int(sys.argv[1])
n = int(sys.argv[2])
MAX_NUM = 50

file.write(str(m) + ' ' + str(n) + '\n')

for _ in range(m):
    for _ in range(n):
        file.write(str(random.randint(0, MAX_NUM)) + ' ')
    file.write(str(random.randint(0, MAX_NUM)) + '\n')

file.close()
\end{lstlisting}


\subsection*{Выводы}

Жадные алгоритмы -- это алгоритмы, которые на каждом этапе выбирается локально оптимальное решение, рассчитывая на то, что и решение всей этой задачи окажется оптимальным. Многие задачи могут быть успешно решены с помощью жадных алгоритмов, причем быстрее, чем другими методами.\newline
В этой задаче я вспомнил линейную алгебру, а точнее метода Гаусса, а так де разобрался с тем, что составление оптимального набора линейно независимых строк -- это задача, которая модет быть решена жадным алгоритмом.

\end{document}