\documentclass[12pt]{article}

\usepackage{listings}
\usepackage{fullpage}
\usepackage{multicol,multirow}
\usepackage{tabularx}
\usepackage{ulem}
\usepackage[utf8]{inputenc}
\usepackage[russian]{babel}
\usepackage{amsmath}
\usepackage{pgfplots}

\begin{document}

\section*{Лабораторная работа №\,9 по курсу дискрeтного анализа: Графы}

Выполнил студент группы М80-308Б-22 МАИ \textit{Кочкожаров Иван}.

\subsection*{Условие}

Краткое описание задачи:
\begin{enumerate}
    \item При помощи алгоритмов на графах разработать решение задачи, определяемой
    своим вариантом
    \item \textit{Вариант:} 6
    \item \textit{Задача:} Поиск кратчайших путей между всеми парами вершин алгоритмом Джонсона
\end{enumerate}

\subsection*{Метод решения}

\begin{enumerate}
    \item Так как алгоритм Дейкстры не умеет работать с отрицательными рёбрами, необходимо на время избавиться от них в нашем графе. Для этого мы добавляем в граф фиктивную вершину \( S \) и строим из неё рёбра с весом \( 0 \) в каждую вершину исходного графа.
    
    \item Для нового графа запускаем алгоритм Беллмана -- Форда, который либо обнаруживает наличие отрицательного цикла в графе и завершает алгоритм, либо возвращает кратчайшие расстояния от фиктивной вершины \( S \) до каждой вершины исходного графа. Суть алгоритма заключается в том, что мы \( V - 1 \) раз проходим по всем рёбрам и релаксируем их, если 
    \[
    d[v] > d[u] + w(u, w).
    \]
    Если на \( V \)-ой итерации происходит ещё одна релаксация, то в графе имеется отрицательный цикл. С помощью этих кратчайших расстояний мы перевзвешиваем рёбра по следующей формуле:
    \[
    \omega'(u, v) = \omega(u, v) + \varphi(u) - \varphi(v).
    \]
    Удаляем фиктивную вершину и запускаем алгоритм Дейкстры для каждой вершины графа, который возвращает кратчайшие расстояния до каждой другой вершины графа. Для преобразования этих расстояний к изначальному графу необходимо применить обратную формулу перевзвешивания:
    \[
    \omega(u, v) = \omega'(u, v) - \varphi(u) + \varphi(v).
    \]
    
    \item Суть алгоритма Дейкстры заключается в том, что в алгоритме поддерживается множество вершин, для которых уже вычислены длины кратчайших путей до них из \( s \). На каждой итерации основного цикла выбирается вершина, не помеченная посещённой, которой на текущий момент соответствует минимальная оценка кратчайшего пути. Вершина добавляется в множество посещённых и производится релаксация всех исходящих из неё рёбер.
\end{enumerate}

\subsection*{Описание программы}

Разделение по файлам, описание основных типов данных и функций. 
\begin{itemize}
    \item
        graph.h
        \begin{lstlisting}[language=C++]
#include <cstddef>
#include <limits>
#include <optional>
#include <queue>
#include <vector>

class Graph {
private:
    struct adjListElem {
        size_t adjNode;
        long long weight;
        adjListElem(size_t adjNode, size_t weight)
            : adjNode(adjNode), weight(weight) {}
        friend bool operator>(const adjListElem &lhs,
                             const adjListElem &rhs) {
            return lhs.weight > rhs.weight;
        }
    };

    std::vector<std::vector<adjListElem>> adjList;
    std::vector<std::vector<long long>> adjMatrix;

    std::vector<long long> Dijkstra(size_t u) {
        size_t n = adjList.size();
        std::vector<long long> distances(n,
            std::numeric_limits<long long>::max());
        distances[u] = 0;
        std::priority_queue<adjListElem, 
                            std::vector<adjListElem>,
                            std::greater<adjListElem>>
            minHeap;
        minHeap.emplace(u, 0);
        std::vector<bool> visited(n, false);
        while (!minHeap.empty()) {
            adjListElem cur = minHeap.top();
            minHeap.pop();
            u = cur.adjNode;
            if (visited[u])
                continue;
            visited[u] = true;
            for (adjListElem el : adjList[u]) {
                size_t v = el.adjNode;
                long long w = el.weight;
                if (u == v)
                    continue;
                if (distances[u] < distances[v] - w) {
                    distances[v] = distances[u] + w;
                    minHeap.emplace(v, distances[v]);
                }
            }
        }
        return distances;
    }

    std::optional<std::vector<long long>>
         BellmanFord(size_t source) {
        size_t n = adjList.size();
        std::vector<long long> distances(n,
                    std::numeric_limits<long long>::max());
        distances[source] = 0;

        for (size_t i = 0; i < n; ++i) {
            for (size_t u = 0; u < n; ++u) {
                for (const adjListElem &edge : adjList[u]) {
                    size_t v = edge.adjNode;
                    long long weight = edge.weight;
                    if (distances[u] < distances[v] - weight) 
                    {
                        if (i == n - 1) {
                            return std::nullopt;
                        }
                        distances[v] = distances[u] + weight;
                    }
                }
            }
        }

        return distances;
    }

public:
    Graph(size_t verticesCount) : adjList(verticesCount + 1),
                                  adjMatrix(
            verticesCount+1,
            std::vector<long long>(verticesCount+1, 
                    std::numeric_limits<long long>::max())) {
        //prepare adjList
        for (size_t i = 1; i < verticesCount + 1; ++i) {
            adjList[0].emplace_back(i, 0);
        }
        //prepare adjMatrix
        for (size_t i = 1; i < verticesCount+1; ++i) {
            adjMatrix[i][i] = 0;
            for (const auto &edge : adjList[i]) {
                adjMatrix[i][edge.adjNode] = edge.weight;
            }
        }
    }

    void AddEdge(size_t u, size_t v, long long weight) {
        adjList[u].emplace_back(v, weight);
    }

    std::optional<std::vector
        <std::vector<long long>>> Johnson() {
        auto potentials = BellmanFord(0);
        if (!potentials) {
            return std::nullopt;
        }
        for (size_t u = 1; u < adjList.size(); ++u) {
            for (auto &el : adjList[u]) {
                auto v = el.adjNode;
                if (el.weight != 
                        std::numeric_limits<long long>::max())
                    el.weight = el.weight + (*potentials)[u] 
                        - (*potentials)[v];
            }
        }
        std::vector<std::vector<long long>> res(
            adjList.size(),
            std::vector<long long>(adjList.size()));
        for (size_t i = 1; i < adjList.size(); ++i) {
            res[i] = Dijkstra(i);
            for (size_t j = 1; j < res[i].size(); ++j) {
                if (res[i][j] != 
                    std::numeric_limits<long long>::max())
                    res[i][j] = res[i][j] + (*potentials)[j] 
                        - (*potentials)[i];
            }
        }
        return res;
    }

    std::vector<std::vector<long long>> FloydWarshall() {
        size_t n = adjList.size();
        auto distance = adjMatrix;

        for (size_t k = 1; k < n; ++k) {
            for (size_t i = 1; i < n; ++i) {
                for (size_t j = 1; j < n; ++j) {
                    if (distance[i][k] !=
                        std::numeric_limits<long long>::max()
                        &&
                        distance[k][j] !=
                        std::numeric_limits<long long>::max() 
                        &&
                        distance[i][j] > 
                            distance[i][k] + distance[k][j]) {
                        distance[i][j] = 
                            distance[i][k] + distance[k][j];
                    }
                }
            }
        }

        return distance;
    }

    Graph() = delete;
};
        \end{lstlisting}

        \item
        main.cpp
        \begin{lstlisting}[language=C++]
#include <cstddef>
#include <iostream>
#include "graph.h"

int main() {
    size_t n, m;
    std::cin >> n >> m;
    Graph graph(n);
    for (size_t i = 0; i < m; ++i) {
        size_t u,v;
        long long w;
        std::cin >> u >> v >> w;
        graph.AddEdge(u, v, w);
    }
    auto allDistances = graph.Johnson();
    if (!allDistances) {
        std::cout << "Negative cycle\n";
        return 0;
    }
    for (size_t i = 1; i < n+1; ++i) {
        for (size_t j = 1; j < n+1; ++j) {
            auto x =(*allDistances)[i][j];
            if (x == std::numeric_limits<long long>::max()) {
                std::cout << "inf ";
            }
            else {
                std::cout << x << ' ';
            }
        }
        std::cout << '\n';
    }
}
        \end{lstlisting}
\end{itemize}

Функции-члены класса:
Приватные:
\begin{enumerate}
    \item std::optional<std::vector<long long>> BellmanFord(size\_t source)
     - функция, которая возвращает вектор, содержащий кратчайшие расстояния от source до всех узлов. Если есть негативный цикл - возвращает std::nullopt.
    \item std::vector<long long> Dijkstra(size\_t u) - возвращает вектор, содержащий кратчайшие расстояния от source до всех узлов. Не работает на графах с отрицательными весами.
\end{enumerate}
Публичные
\begin{enumerate}
    \item Graph(size\_t verticesCount) - создает граф в виде списка смежности и матрицы смежности.
    \item void AddEdge(size\_t u, size\_t v, long long weight) - добавляет ребро в граф.
    \item std::optional<std::vector<std::vector<long long>>> Johnson() - возвращает матрицу кратчайших расстояний для графа.
    \item std::optional<std::vector<std::vector<long long>>> - FloydWarshall() - возвращает матрицу кратчайших расстояний для графа (реализован для сравнения с предыдущим алгоритмом)
\end{enumerate}

\subsection*{Дневник отладки}

Во время реализации я столкнулся с проблемами:
\begin{enumerate}
    \item Проблема с переполнением. При вычислении потенциалов Джонсона переменные с бесконечностями обрабатывались некорректно.
    \item Проблема с RE на чекере. Проблема возникала из-за того, что я возвращал кол ошибки 1 при обнаружении отрицательного цикла. После замены кода на 0 код прошел чекер.
\end{enumerate}

\subsection*{Тест производительности}

\begin{lstlisting}[language=C++]
#include <chrono>
#include <random>
#include <iostream>
#include "graph.h"

int main(int argc, char* argv[]) {
    if (argc < 3) {
        std::cerr << "Usage: " << argv[0]
         << " <number_of_nodes> <number_of_edges>\n";
        return 1;
    }

    size_t n = std::strtoull(argv[1], nullptr, 10);
    size_t m = std::strtoull(argv[2], nullptr, 10);

    Graph graph(n);

    std::mt19937 rng(42);
    std::uniform_int_distribution<size_t> nodeDist(1, n);
    std::uniform_int_distribution<long long> weightDist(0, 1000);

    for (size_t i = 0; i < m; ++i) {
        size_t u = nodeDist(rng);
        size_t v = nodeDist(rng);
        long long w = weightDist(rng);
        graph.AddEdge(u, v, w);
    }

    auto start = std::chrono::high_resolution_clock::now();
    auto johnsonResult = graph.Johnson();
    auto end = std::chrono::high_resolution_clock::now();
    auto johnsonDuration = std::chrono::duration_cast
            <std::chrono::milliseconds>(end - start).count();

    if (!johnsonResult) {
        std::cout << 
        "Negative cycle detected in Johnson algorithm\n";
    } else {
        std::cout 
        << "Johnson algorithm completed in " 
        << johnsonDuration << " ms\n";
    }

    start = std::chrono::high_resolution_clock::now();
    auto floydWarshallResult = graph.FloydWarshall();
    end = std::chrono::high_resolution_clock::now();
    auto floydDuration = std::chrono::duration_cast
        <std::chrono::milliseconds>(end - start).count();

    std::cout << "Floyd-Warshall algorithm completed in " 
    << floydDuration << " ms\n";

    return 0;
}
\end{lstlisting}

Количество ребер будет фиксированным и равным 4000. Будем изменять количество вершин

\begin{tikzpicture}
    \begin{axis} [
        ymin = 0,
        ylabel=$ms$,
        xlabel=$N$,
    ]
    \addplot coordinates {
         (100, 44) (200, 93) (300, 171) (400, 261) (500, 377)
    };
    \addplot coordinates {
         (100, 22) (200, 148) (300, 465) (400, 1020) (500, 1959)
    };
    \end{axis}
\end{tikzpicture}

Красный граф - алгорит Флойда-Уоршелла, синий - Джонсона.
Таким образом видим, что алгоритм Джонсона отлично подходит для разреженного графа. А Флойд-Уоршелл может обгонять его на полных графах, что видно в начале (у графа с 100 вершин максимальное количество ребер - 4950).

Сложность алгоритма Джонсона - $O(n(n+m)logn)$, т.к. мы используем алгоритм Дейкстры со сложностью $O((n+m)logn)$ $n$ раз.


\subsection*{Выводы}
В ходе выполнения лабораторной работы я изучил алгоритм Джонсона и применил его для решения задачи нахождения кратчайших путей между всеми парами вершин в графе. Полученные знания позволили успешно реализовать и проверить работу алгоритма на практике.




\end{document}